% https://en.wikibooks.org/wiki/LaTeX/Curriculum_Vitae
\documentclass[10pt,a4paper,roman]{moderncv}

%% ModernCV themes
\moderncvstyle{banking}
\moderncvcolor{blue}
\renewcommand{\familydefault}{\sfdefault}
\nopagenumbers{}

%% Character encoding
\usepackage[utf8]{inputenc}

%% Adjust the page margins
\usepackage[scale=0.75]{geometry}
\usepackage{url}


\usepackage{multibib}
\newcites{proceedings,talk}{{International Conference (peer-reviewed)},{Talk/Tutorial}}

%% Personal data
\firstname{Shigeki}
\familyname{Karita}
%\title{Resumé} 
\address{Osaka, Japan}
%\mobile{+81~(80)~6329~7098} 
% \phone{+2~(345)~678~901} 
% \fax{+3~(456)~789~012} 
\email{karita@ieee.org} 
\social[github]{ShigekiKarita}
\homepage{https://karita.xyz} 
%\homepage{https://scholar.google.com/citations?user=enV4FrIAAAAJ}
\extrainfo{updated \today} 
%\photo[64pt][0.4pt]{picture}
%\quote{Some quote (optional)}



% bibliography adjustements (only useful if you make citations in your resume, or print a list of publications using BibTeX)
%   to show numerical labels in the bibliography (default is to show no labels)
\makeatletter\renewcommand*{\bibliographyitemlabel}{\@biblabel{\arabic{enumiv}}}\makeatother

%%------------------------------------------------------------------------------
%% Content
%%------------------------------------------------------------------------------
\begin{document}
\makecvtitle

%%%%%%%%%%%%%%%%%%%%%%%%%%%%%%%%%%%%%%%%%%%%%%%%%%%
\section{Work Experience}
%%%%%%%%%%%%%%%%%%%%%%%%%%%%%%%%%%%%%%%%%%%%%%%%%%%

\cventry{April 2016 - present}{Research Scientist}{NTT Communication Science Laboratories}{Kyoto, Japan}{ }{Reseach and develop automatic speech recognition (ASR) and machine learning systems in C++, CUDA and Python at NTT.  Mostly working with Dr. Atsunori Ogawa, Dr. Marc Delcroix at NTT, and Dr. Shinji Watanabe at JHU.}  % arguments 3 to 6 can be left empty

\cventry{April 2015 - March 2016}{Teaching Assistant}{Graduate School of Engineering, Osaka University}{Osaka, Japan}{ }{Wrote several textbooks for "Programming Exercise" class that covers signal processing, statistical analysis and tiny compiler in C++ and Java, and advised students in the class with Assistant Prof. Kazuaki Nakamura.}  % arguments 3 to 6 can be left empty

\cventry{September 2015 - March 2015}{Research Intern}{NTT Communication Science Laboratories}{Kyoto, Japan}{ }{Implemented a convolutional neural network (CNN) acoustic model from scratch in C++, CUDA and OpenCL, and published CNN-based reverberant robust ASR paper at ICASSP 2016~\cite{Yoshioka2015} with Dr. Takuya Yoshioka at NTT.}

%%%%%%%%%%%%%%%%%%%%%%%%%%%%%%%%%%%%%%%%%%%%%%%%%%%
\section{Education}
%%%%%%%%%%%%%%%%%%%%%%%%%%%%%%%%%%%%%%%%%%%%%%%%%%%
\cventry{April 2014 - March 2016}{M.Eng.}{Graduate School of Engineering, Osaka University}{Osaka, Japan}{ \textit{Electronic and Information Engineering} }{Advisor: Prof. Noboru Babaguchi}  % arguments 3 to 6 can be left empty

\cventry{April 2010 - March 2014}{B.Eng.}{School of Engineering, Osaka University}{Osaka, Japan}{ \textit{Electronic and Information Engineering} }{Advisor: Prof. Noboru Babaguchi}  % arguments 3 to 6 can be left empty

\section{Projects}
\cvitem{Automatic Speech Recognition at NTT}{Research Scientist}
\cvlistitem{This project aims to research and develop automatic speech recognition (ASR) systems at NTT.}
\cvlistitem{My research interest includes various extensions for acoustic models and end-to-end models in ASR, for example, noise robust ASR~\cite{Karita2017}, far-field ASR~\cite{Yoshioka2015}, sequential training~\cite{karita_icassp2018}, semi-supervised training~\cite{Karita_interspeech2018,karita_icassp2019}.\newline}

\cvitem{ESPnet: end-to-end speech processing toolkit}{Core Developer} % {\url{https://github.com/espnet/espnet}}
\cvlistitem{This project aims to be an open source state-of-the-art platform for end-to-end speech processing~\cite{Watanabe2018}.}
\cvlistitem{I mainly develop and maintain ASR and TTS pytorch backend, and continuous integration (CI) for unittesting and sphinx documentation. For example, my early PR made ESPnet 3-4 times faster. \url{https://github.com/espnet/espnet/pull/17}}
% As my weekend projects at GitHub, I have developed various machine learning frameworks, e.g., GPU-based autograd tensor library, SVM, decision trees.


\section{Skills}
\cvitem{Research}{ {Speech and signal processing, Machine learning, High performance computing.} }
\cvitem{Programming}{ {C++, CUDA, OpenCL, D, Python, Java, Scala, Rust} }
\cvitem{Language}{Japanese (native), English (full professional), German (limited, read-only)}

%% A publications list
\section{Publications}

also see  \url{https://scholar.google.com/citations?user=enV4FrIAAAAJ}
\newline

% TODO patent, tutorial at INTERSPEECH, staff at ASRU

\nociteproceedings{*}
\bibliographystyleproceedings{ieeetr}
\bibliographyproceedings{proceedings}

\nocitetalk{*}
\bibliographystyletalk{ieeetr}
\bibliographytalk{talk}

\subsection{Patents}
Contributed to more than 5 patents at NTT

%\citesinthissection{5}% There are 3 cites in this section + 2 above
% \nocite{karita_icassp2019}
% \printbibliography[type=article,title={Papers}]


\end{document}